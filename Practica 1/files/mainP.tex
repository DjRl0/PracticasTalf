\documentclass[fleqn, 10pt]{article}

% Paquetes necesarios
\usepackage[utf8]{inputenc}
\usepackage[spanish]{babel}
\usepackage{amsthm, amsmath}
\usepackage{nccmath} %Para centrar ecuaciones
\usepackage{graphicx}
\usepackage{enumitem}

% Personalizo mi alfabeto
\DeclareMathAlphabet{\pazocal}{OMS}{zplm}{m}{n}
\newcommand{\Lb}{\pazocal{L}}

% Definimos los entornos para definiciones, teoremas, etc...
\theoremstyle{plain}
\newtheorem{proposicion}{Proposición}

\theoremstyle{definition}
\newtheorem{definition}{Definición}[section]
\newtheorem{example}{Ejemplo}[section]

%Definimos el título
\title{Teoría de Autómatas y Lenguajes Formales\\[.4\baselineskip]Práctica 1: Latex y expresiones regurales}
\author{Nombre, Apellidos}
\date{\today}

%Comienzo del documento
\begin{document}

%Generamos el título
\maketitle

\section{Expresiones regulares}

Las \textit{expresiones regulares} ($\mathcal{R}$) son un método de representación de
lenguajes. Aunque su potencia expresiva es limitada, haciendo que sólo los
lenguajes regulares puedan representarse con ellas, tienen la virtud de una gran
sencillez en su formulación.


\begin{definition}[\textbf{\textit{Aplicación $\Lb$}}]\label{def:aplicL}
	La aplicación $\Lb$ establece una relación formal entre las expresiones regulares y los lenguajes que éstos representan, definiéndose como sigue:
  \begin{ceqn}	%Para definir ecuación centrada en el texto
    \begin{align*} %Ecuación multilínea con alineamiento personalizado (split y align)
    \Lb: \mathcal{R} \rightarrow 2^{\Sigma^*}\\ 
    r \rightarrow \Lb(r)
    \end{align*} 
  \end{ceqn} 
  
\begin{enumerate}[label=\alph{enumi})]
  \item $\Lb(\emptyset)=\emptyset$ 
  \item $\Lb(a)=\{a\} \, \forall a\in\Sigma$ 
  \item Si $\alpha,\beta \in \mathcal{R}$ entonces $\Lb((\alpha\beta))=\Lb(\alpha)\Lb(\beta)$
  \item Si $\alpha,\beta \in \mathcal{R}$ entonces $\Lb((\alpha+\beta))=\Lb(\alpha)\cup \Lb(\beta)$
  \item Si $\alpha \in \mathcal{R}$ entonces $\Lb(\alpha^*)=\Lb(\alpha)^*$
\end{enumerate}

\end{definition}

\subsection{Propiedades de las expresiones regulares}
\begin{proposicion}
Si $\alpha,\beta,\gamma$ son expresiones regulares entonces se cumple:
  \begin{equation}
  (\alpha+\beta)\gamma=\alpha\gamma+\beta\gamma
  \end{equation}
\end{proposicion}
\begin{proof}
Usando las reglas de la definición \ref{def:aplicL} tenemos que:
\begin{multline*}
\Lb(((\alpha+\beta)\gamma))=\Lb((\alpha+\beta))\Lb(\gamma)=(\Lb(\alpha)\cup \Lb(\beta))\Lb(\gamma)=\Lb(\alpha)\Lb(\gamma)\cup \Lb(\beta)\Lb(\gamma)=
\end{multline*}
\end{proof}

\begin{example}
Consideremos $L=\{w\in \{a,b\}^* : w \textnormal{ no termina en } ab\}$. Un expresión regular que genera L es: \\
ESCRIBIR SOLUCIÓN COMO ECUACIÓN CENTRADA NO NUMERADA

\end{example}


\end{document}