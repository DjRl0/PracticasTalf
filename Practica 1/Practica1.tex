\documentclass{article}
% pre\'ambulo

\usepackage{lmodern}
\usepackage[T1]{fontenc}
\usepackage[spanish,activeacute]{babel}
\usepackage{mathtools}
\usepackage{amsmath}
\usepackage[a5paper,margin=1in,top=15mm,bottom=15mm,landscape]{geometry} 

\title{\textbf{Pr\'actica 1}}
\author{\\Diego Jes\'us Romero Luque}
\date{\today}


\begin{document}
\maketitle
\pagebreak

% cuerpo del documento
\begin{enumerate}
    \item \textbf{R$^3$ of R = $ \{(1,1),(1,2),(2,3),(3,4)\}$}\\
        R \space \space =
        $\begin{pmatrix}
            1 & 0 & 0 & 0\\
            1 & 0 & 0 & 0\\
            0 & 1 & 0 & 0\\
            0 & 0 & 1 & 0
        \end{pmatrix}$, 
        R$^3$ = R x R x R $\rightarrow$
        R$^3$ =
        $\begin{pmatrix}
            1 & 0 & 0 & 0\\
            1 & 0 & 0 & 0\\
            0 & 1 & 0 & 0\\
            0 & 0 & 1 & 0
        \end{pmatrix}$ x
        $\begin{pmatrix}
            1 & 0 & 0 & 0\\
            1 & 0 & 0 & 0\\
            0 & 1 & 0 & 0\\
            0 & 0 & 1 & 0
        \end{pmatrix}$ x
        $\begin{pmatrix}
            1 & 0 & 0 & 0\\
            1 & 0 & 0 & 0\\
            0 & 1 & 0 & 0\\
            0 & 0 & 1 & 0
        \end{pmatrix}$,\\\\\\
        R$^3$ = 
        $\begin{pmatrix}
            1 & 0 & 0 & 0\\
            1 & 0 & 0 & 0\\
            1 & 0 & 0 & 0\\
            0 & 1 & 0 & 0
        \end{pmatrix}$ x
        $\begin{pmatrix}
            1 & 0 & 0 & 0\\
            1 & 0 & 0 & 0\\
            0 & 1 & 0 & 0\\
            0 & 0 & 1 & 0
        \end{pmatrix}$ =
        $\begin{pmatrix}
            1 & 0 & 0 & 0\\
            1 & 0 & 0 & 0\\
            1 & 0 & 0 & 0\\
            1 & 0 & 0 & 0
        \end{pmatrix}$
        \\\\
        El resultado es: R$^3$ = $\{(1,1),(1,2),(1,3),(1,4)\}$ y podemos comprobarlo con el script 
        \textit{powerrelation.m} :
        \begin{verbatim}
            octave:8> powerrelation({['1','1'],['1','2'],['2','3'],['3','4']},3)
            ans =
                {
                 [1,1] = 11
                 [1,2] = 12
                 [1,3] = 13
                 [1,4] = 14
                }

        \end{verbatim}
        

    \pagebreak
    \item \textbf{Within the folder "files", find a TEX file in whose content appears 
           the string "\textbackslash usepackage\{amsthm, amsmath\}".}
        \begin{verbatim}
            $ cd files/
            $ grep -n -F "\usepackage{amsthm, amsmath}" *.tex
            mainP.tex:6:\usepackage{amsthm, amsmath} 
        \end{verbatim}
        \textbf{Consideremos $L=\{w\in \{a,b\}^* : w \textnormal{ no termina en } ab\}$. Un expresión 
         regular que genera L es:} \\
            \begin{equation*}
                L=\{w \in \{a,b\}^* : w=(a+b)^*(aa+ba+bb)\}
            \end{equation*}
    
    \end{enumerate}
\end{document}